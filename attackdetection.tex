% -*-  mode: TeX-PDF-mode;  -*-
\documentclass[compress]{beamer}
\mode<presentation>

\usetheme{metropolis}
\graphicspath{{./}{images/}}

\definecolor{darkgreen}{rgb}{0, .6, 0}
\definecolor{darkred}{rgb}{.75,0,0}

% include packages
\usepackage[english]{babel}
\usepackage[utf8]{inputenc}
\usepackage{lmodern}
\usepackage{scalefnt}
\usepackage{attrib}
\usepackage{multicol}
\usepackage{mathtools}
\usepackage[compatibility=false]{caption}
\usepackage{booktabs}
\usepackage{csquotes}
\usepackage{siunitx}
\usepackage{graphicx}
\usepackage{tikz}
\usepackage{standalone}
\usepackage{url}
\usepackage{multimedia}
\usepackage{hyperref}
\usepackage{url}
\usepackage{braket}
\usepackage{verbatim}
\usepackage{tcolorbox}
\usepackage{relsize}
\usepackage{subcaption}
\usepackage{allrunes}
\usepackage{siunitx}
\usepackage{textcomp}
\usepackage[normalem]{ulem}
\usetikzlibrary{shapes.geometric,shapes,positioning,shapes.symbols,arrows,calc,fit,external}

\tcbuselibrary{listingsutf8,breakable,theorems,skins}

\captionsetup[figure]{labelformat=empty}
\captionsetup[table]{labelformat=empty}
\setbeamertemplate{caption}{\insertcaption}

\setbeamertemplate{footline}[text line]{%
%LEO: I need footnotes.%\parbox{\linewidth}{\vspace*{-12pt}\hfill{}JonathanJogenfors\hfill{}\texttt{jonathan.jogenfors@liu.se}\hfill\insertframenumber/\inserttotalframenumber}}
\parbox{\linewidth}{\vspace*{-12pt}\hfill{}\hfill{}\hfill\insertframenumber/\inserttotalframenumber}}
\setbeamertemplate{navigation symbols}{}


\setbeamerfont{subtitle}{series=\normalfont}

\title{Attack Detection}
\subtitle{}
\author{Jonathan Jogenfors\\ Leonardo Iwaya}
\date{2016-09-26}

\newtcbinputlisting[]{\listing}[1][]{%
    listing only,
    fontupper=\huge,
}
\begin{document}
\small
\frame{\titlepage}
\begin{frame}{Introduction}
    \begin{columns}
        \begin{column}{0.8\textwidth}
    \begin{itemize}
        \item About us
    \end{itemize}
        \end{column}
        \begin{column}{0.2\textwidth}  %%<--- here
            \begin{center}

                \includegraphics[width=\textwidth]{jonathan.jpg}


            \end{center}
        \end{column}
    \end{columns}
\end{frame}
\begin{frame}
    \begin{itemize}
        \item Model of information security: Prevention, Detection, Reaction
            (PRD)
            \begin{itemize}
                \item Prevention: Difficult because attacker has the advantage.
                    Large attack surface.


                \item
                    Reaction: Too late!
                \item
                    Detection: What our papers are all about
            \end{itemize}
        \item

    \end{itemize}
\end{frame}
\begin{frame}{When prevention fails}

\end{frame}
\begin{frame}{Anomaly detection}

\end{frame}
\begin{frame}{Intrusion Detection Systems \& Honeypots}
\begin{figure}
 \centering
 \includegraphics[width=1.0\textwidth]{./images/ids-and-honeypot.png}
 % ids-and-honeypot.png: 0x0 pixel, 300dpi, 0.00x0.00 cm, bb=
 \caption{Source: http://www.computerworld.com/article/2592425/lan-wan/intrusion-detection.html}
 \label{fig:ids-and-honeypot}
\end{figure}

\end{frame}
\begin{frame}{Control Theory \& Cybernetics}
    \textbf{Control systems}: measure, compare, compute and correct.
    \begin{figure}
    \centering\tiny
    \includegraphics[width=1.0\textwidth]{./images/control-theory.png}
    % control-theory.png: 0x0 pixel, 300dpi, 0.00x0.00 cm, bb=
    \caption{\textbf{Feedback loop} to control the behavior of a system by comparing its output to a desired value, and applying the difference as an error signal to dynamically change the output so it is closer to the desired output.}
    \label{fig:control-theory}
    \end{figure}
\end{frame}
\begin{frame}{Papers}
    \textbf{Motivation}
    \begin{itemize}
     \item Papers \cite{pasqualetti2013attack} and \cite{genge2015system} were \textbf{selected} based on their \textbf{relevance} to the theme of \textit{``attack detection methods''} for \textit{``critical infrastructures''}. 
     \item Paper \cite{vasilo2016multi} was in the course's reading list.

    \end{itemize}
    \begin{table}\tiny
	\begin{tabular}{|l|l|l|l|l|}
	\hline
	\textbf{Paper} & \textbf{Year} & \textbf{CI Sub-area} & \textbf{Citations} & \textbf{Journal IF}\\ 
	\hline \hline
	Pasqualetti et al \cite{pasqualetti2013attack} & 2013 & Attack Detection & 210 & 2.777\\
	Genge et al \cite{genge2015system} & 2015 & Attack Prevention \& Detection & 12 & 1.351\\
	Vasilomanolakis et al \cite{vasilo2016multi} & 2016 & Attack Detection & 1 & n.a \\
	\hline
	\end{tabular}
    \end{table}
\end{frame}
\begin{frame}{Attack detection and identification in cyberphysical systems
    (2013)}

\end{frame}
\begin{frame}{A system dynamics approach for assessing the impact of cyber attacks on CI (2015)}
    \textbf{Aim \& Contribution}
    \begin{itemize}
     \item To \textbf{identify} and \textbf{rank} assets in complex, large-scale and heterogeneous CIs.
     \item \textbf{Cyber Attack Impact Assessment} (CAIA) methodology that helps system admins to understand:
     \begin{enumerate}
      \item How cyber attacks affect the normal functioning of physical processes?
      \item What cyber assets would cause the most negative impact if compromised?
     \end{enumerate}
    \end{itemize}
\end{frame}
\begin{frame}{A system dynamics approach for assessing the impact of cyber attacks on CI (2015)}
    \textbf{CAIA Methodology}
    \begin{figure}
      \centering
      \includegraphics[width = 0.75\textwidth]{./images/caia-method.png}
      % caia-method.jpg: 0x0 pixel, 300dpi, 0.00x0.00 cm, bb=
      % \caption{CAIA Methodology.}
      \label{fig:caia-method}
    \end{figure}
\end{frame}
\begin{frame}{A system dynamics approach for assessing the impact of cyber attacks on CI (2015)}
    \textbf{Experiments \& Comparisons}
    \begin{itemize}
     \item First, the \textbf{basic functioning} of CAIA is demonstrated using IEEE 14-bus electric grid model.
     \item Second, CAIA's \textbf{scalability} is proven by using attack scenarios in the context of IEEE 300-bus electric grid model.
     \item Third, CAIA's \textbf{cross-sector applicability} is evaluated using Tennessee Eastman chemical process system.
     \item The methodology was also \textbf{compared} with other approaches (i.e., graph-theoretic and electrical centrality metric techniques).
    \end{itemize}
\end{frame}
\begin{frame}{A system dynamics approach for assessing the impact of cyber attacks on CI (2015)}
    \begin{columns}
     \begin{column}{0.55\textwidth}
      \begin{figure}
      \centering
      \includegraphics[width=1.0\textwidth]{./images/ieee-14-bus.png}
      % ieee-14-bus.png: 0x0 pixel, 300dpi, 0.00x0.00 cm, bb=
      \label{fig:ieee-14-bus}
      \end{figure}
     \end{column}
     \begin{column}{0.45\textwidth}
      \begin{figure}
      \centering
      \includegraphics[width=1.0\textwidth]{./images/caia-param.png}
      % caia-param.png: 0x0 pixel, 300dpi, 0.00x0.00 cm, bb=
      \label{fig:caia-param}
      \end{figure}
     \end{column}
    \end{columns}
\end{frame}
\begin{frame}{A system dynamics approach for assessing the impact of cyber attacks on CI (2015)}
    \begin{columns}
     \begin{column}{0.55\textwidth}
      \begin{figure}
      \centering
      \includegraphics[width=1.0\textwidth]{./images/ieee-14-bus.png}
      % ieee-14-bus.png: 0x0 pixel, 300dpi, 0.00x0.00 cm, bb=
      \label{fig:ieee-14-bus-2}
      \end{figure}
     \end{column}
     \begin{column}{0.45\textwidth}
      \begin{figure}
      \centering
      \includegraphics[width=1.0\textwidth]{./images/caia-param-impact.png}
      % caia-param-impact.png: 0x0 pixel, 300dpi, 0.00x0.00 cm, bb=
      \label{fig:caia-param-impact}
      \end{figure}
     \end{column}
    \end{columns}
\end{frame}
\begin{frame}{A system dynamics approach for assessing the impact of cyber attacks on CI (2015)}
    \begin{figure}
    \centering
    \includegraphics[width=1.0\textwidth]{./images/stealthy-attack-sequence.png}
    % stealthy-attack-sequence.png: 0x0 pixel, 300dpi, 0.00x0.00 cm, bb=
    \label{fig:stealthy-attack-sequence}
    \end{figure}
\end{frame}
\begin{frame}{A system dynamics approach for assessing the impact of cyber attacks on CI (2015)}
    \begin{columns}
     \begin{column}{0.4\textwidth}
      \begin{figure}
      \centering
      \includegraphics[width=1.0\textwidth]{./images/caia-impact-stealthy.png}
      % caia-impact-stealthy.png: 0x0 pixel, 300dpi, 0.00x0.00 cm, bb=
      \label{fig:ieee-14-bus}
      \end{figure}
     \end{column}
     \begin{column}{0.6\textwidth}
      \begin{figure}
      \centering
      \includegraphics[width=1.0\textwidth]{./images/stealthy-operator-view.png}
      % stealthy-operator-view.png: 0x0 pixel, 300dpi, 0.00x0.00 cm, bb=
      \label{fig:stealthy-operator-view}
      \end{figure}
     \end{column}
    \end{columns}
\end{frame}
\begin{frame}{A system dynamics approach for assessing the impact of cyber attacks on CI (2015)}
    \textbf{Limitations}
    \begin{itemize}
     \item CAIA helps to identify and rank assets given specific interventions (e.g., an attack)
     \item Which interventions are relevant to test (?), and, how to protect the assets after generating the impact matrix (?) are open questions; out of the paper's scope.
     \item Obvious Note: the knowledge of impact matrices would be definitely valuable to attackers(!); as any risk assessment information.
     \item Seems hard to reproduce since no detailed information is given about the simulations; plus, no source code.
    \end{itemize}
\end{frame}
\begin{frame}{Multi-stage Attack Detection and Signature Generation with ICS Honeypots (2016)}
    Aim \& Contribution
    \begin{itemize}
     \item HosTaGe: honeypot for detecting multi-stage attacks in ICS networks.
     \item Honeypot extension with capabilities of ICS protocols, i.e., Modbus, S7, SNMP, HTTP, Telnet, SMB and SMTP.
     \item Basic functions:
     \begin{enumerate}
      \item notify the network administrators;
      \item produce an attack signature;
      \item forward the signature to the internal IDSs.
     \end{enumerate}
    \end{itemize}
\end{frame}
\begin{frame}{Multi-stage Attack Detection and Signature Generation with ICS Honeypots (2016)}
    Expermients \& Comparisons
    \begin{itemize}
     \item HosTaGe was compared with ``CONPOT ICS/SCADA Honeypot''~\footnote{http://conpot.org/}
     \item Criteria:
     \begin{enumerate}
      \item ability to not be evade (i.e., be perceived by attackers);
      \item ability to detect multi-stage attacks;
      \item ability to generate valid signatures for Bro IDS~\footnote{https://www.bro.org/}.
     \end{enumerate}
    \end{itemize}
\end{frame}
\begin{frame}{Multi-stage Attack Detection and Signature Generation with ICS Honeypots (2016)}
    Formal Model - Extended Finite State Machine (EFSM)
    \begin{figure}
    \centering
    \includegraphics[width=1.0\textwidth]{./images/hostage-efsm.png}
    % hostage-efsm.png: 0x0 pixel, 300dpi, 0.00x0.00 cm, bb=
    \label{fig:hostage-efsm}
    \end{figure}
\end{frame}
\begin{frame}{Multi-stage Attack Detection and Signature Generation with ICS Honeypots (2016)}
    Formal Model - Extended Finite State Machine (EFSM)
    \begin{itemize}
     \item Detection Mechanism
     \begin{enumerate}
      \item Single-Protocol Level Detection (SPLD)
      \item Multi-Stage Level Detection (MSLD)
      \item Payload Level Detection (PLD)
     \end{enumerate}
     \item Time window ($tw$) determines whether an attack should be mapped as SPLD or MSLD
    \end{itemize}
\end{frame}
% \begin{frame}{Multi-stage Attack Detection and Signature Generation with ICS Honeypots (2016)}
%     Example - Detection of Stuxnet
%     \begin{figure}
%     \centering
%     \includegraphics[width=1.0\textwidth]{./images/hostage-stuxnet-detection.png}
%     % hostage-stuxnet-detection.png: 0x0 pixel, 300dpi, 0.00x0.00 cm, bb=
%     \label{fig:hostage-stuxnet-detection}
%     \end{figure}
% \end{frame}
\begin{frame}{Multi-stage Attack Detection and Signature Generation with ICS Honeypots (2016)}
    Example - Signature Generation
    \begin{itemize}
     \item Automatically generate signature for well-known Metasploit script\footnote{No further information given by the authors...} for Modbus services identification.
    \end{itemize}
    \begin{figure}
    \centering
    \includegraphics[width=1.0\textwidth]{./images/hostage-signature.png}
    % hostage-signature.png: 0x0 pixel, 300dpi, 0.00x0.00 cm, bb=
    \label{fig:hostage-signature}
    \end{figure}
\end{frame}
\begin{frame}{Multi-stage Attack Detection and Signature Generation with ICS Honeypots (2016)}
    Comparison - Honeypot x CONPOT
    \begin{itemize}
     \item Controlled environment, no firewalls, 8 to 12 weeks, probing by Shodan\footnote{https://www.shodan.io/}.
    \end{itemize}
    \begin{columns}
     \begin{column}{0.55\textwidth}
      \begin{figure}
      \centering
      \includegraphics[width=1.0\textwidth]{./images/hostage-conpot-comparison.png}
      % hostage-conpot-comparison.png: 0x0 pixel, 300dpi, 0.00x0.00 cm, bb=
      \label{fig:hostage-conpot-comparison}
      \end{figure}
     \end{column}
     \begin{column}{0.45\textwidth}
      \begin{figure}
      \centering
      \includegraphics[width=1.0\textwidth]{./images/hostage-ip-sources.png}
      % hostage-ip-sources.png: 0x0 pixel, 300dpi, 0.00x0.00 cm, bb=
      \label{fig:hostage-ip-sources}
      \end{figure}
     \end{column}
    \end{columns}
\end{frame}
\begin{frame}{Multi-stage Attack Detection and Signature Generation with ICS Honeypots (2016)}
    \textbf{Limitations}
    \begin{itemize}
     \item The evaluation of multi-stage signature generation was rather shallow.
     \item Shodan's probes were not explained in details, i.e., how Shodan detect a honeypot?
    \end{itemize}
    More info about HosTaGe can be found at Darmstad's research group website\footnote{https://www.tk.informatik.tu-darmstadt.de/de/research/secure-smart-infrastructures/hostage/}.
\end{frame}

\begin{frame}{Discussions}
      %%Input good questions here.
\end{frame}

\begin{frame}[allowframebreaks]{References}\tiny{
\def\newblock{}
\bibliographystyle{plain}
\bibliography{./references.bib}}
\end{frame}
\end{document}
